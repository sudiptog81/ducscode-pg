\documentclass{article}
\usepackage[english]{babel}
\usepackage[a4paper,top=2cm,bottom=2cm,left=3cm,right=3cm,marginparwidth=1.75cm]{geometry}
\usepackage{amsmath}
\usepackage{amssymb}
\usepackage[affil-it]{authblk}
\usepackage{graphicx}
\usepackage[english]{isodate}
\usepackage[colorlinks=true, allcolors=blue]{hyperref}
\usepackage{algorithm2e}
\usepackage{csvsimple}
\usepackage{longtable}
\usepackage{hyperref}

\title{Lab Report 1}
\author{Sudipto Ghosh\\\it{MCSC101 -- M. Sc. Semester I}}
\affil{Department of Computer Science\\University of Delhi}
\date{\printdayoff December 15, 2022}

\begin{document}
\maketitle

\section{Insertion Sort}
\RestyleAlgo{ruled}
\SetKwComment{Comment}{/* }{ */}
\begin{algorithm}
\caption{Insertion Sort}
\label{alg:one}
\KwData{Array $A[1..n]$, Array Size $n$}
\For{$i$ = $2..n$}{
    $Insert(A[1...i-1],i)$ \Comment{this function searches for an appropriate
    location $j$ to insert $A[i]$ in $A[1...i-1]$ so that $A[1...i]$ is sorted, and inserts $A[i]$ in the $j^{th}$ location}
}
\end{algorithm}
\RestyleAlgo{ruled}
\SetKwComment{Comment}{/* }{ */}
\begin{algorithm}
\caption{Insert}
\label{alg:two}
\KwData{Array $A[1..i]$, Index $i$}
$j \gets i - 1$\;
$key \gets A[i]$\;
\While{$j \geq 1$}{
    \Comment{compare $A[j]$ and $key$}
    \eIf{$A[j] > key$}{
        $A[j + 1] \gets A[j]$\;
        $j \gets j - 1$\;
    }{
        break\;
    }
}
\Comment{insert $A[i]$ in the $j^{th}$ location}
$A[j + 1] \gets key$\;
\end{algorithm}
\section{Solutions}
\subsection{Problem 1}
The possible locations that $A[i]$ may take in the $i^{th}$ iteration are $1...i$ where $1 \leq i \leq n$.

\subsection{Problem 2}
The number of comparisons performed by Insertion Sort to insert $A[i]$ in location $j$ is given by $i - j + 1$ or $i - j$, which is supported by comparing the observed values and the value of the expression for all values of $i$ and $j$ as shown in \hyperref[sec:data]{this table}.

\subsection{Problem 3}
On running Insertion Sort for all possible permutations of 1, 2, 3, 4, we observe the following behaviour.\\~\\
For $i=1$, all $A[i]$ get placed at location $j=1$, with probability 1.\\~\\
For $i=2$, we have
\csvautolongtable[respect all]{occ2.csv}
We observe that the probabilities of $A[i]$ to get placed in $j=1$ and $j=2$ are equal and given by $1/2$.\\~\\
For $i=3$, we have
\csvautolongtable[respect all]{occ3.csv}
We observe that the probabilities of $A[i]$ to get placed in $j=1$, $j=2$ and $j=3$ are equal and given by $1/3$.\\~\\
For $i=4$, we have
\csvautolongtable[respect all]{occ4.csv}
We observe that the probabilities of $A[i]$ to get placed in $j=1$, $j=2$, $j=3$ and $j=4$ are equal and given by $1/4$.\\\\
We can thus say that the probability that $A[i]$ will be inserted at the $j^{th}$ location is $1/i$. $\blacksquare$
\subsubsection{Part (a)}
For every $i=2..4$, we report the total number of comparisons done to insert $A[i]$ to the $j^{th}$ location and compute the average number of comparisons over all possible permutations. The observations are recorded in the table below.
\csvautolongtable[respect all]{comps.csv}
\subsubsection{Part (b)}
We construct a 2D-table with $i$ in rows and $j$ in columns where each $P[i,j]$ represents the probability that $A[i]$ will be inserted in location $j$ in the $i^{th}$ iteration.
\csvautolongtable[respect all]{probs.csv}
\subsubsection{Part (c)}
TODO.
\subsection{Problem 4}
TODO.
\section{Raw Data}
\label{sec:data}
\csvautolongtable[respect all]{sort.csv}
\end{document}